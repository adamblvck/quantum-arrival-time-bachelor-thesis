\section*{ABSTRACT}

\textbf{Canonical quantum mechanics focuses on measurement outcomes, while alternative formulations like Bohmian mechanics describe the particle's path from A to B, while yielding the same predictions. In experiments like the double-slit, the concept of "travel time" or "arrival time" lacks precision due to issues with initial conditions. However, free-fall experiments with quantum particles on opaque barriers might resolve this. This study introduces an analysis of such setups, calculating arrival times using Bohmian probability currents at the detector location. The resulting distributions reveal band formation, tunneling resonances, and other features, suggesting further theoretical and experimental exploration. The study also compares tunneling times predicted by the Larmor clock for spin-$\frac{1}{2}$ particles and Bohmian mechanics. Both theories agree with square barriers, but show differences with Gaussian barriers, highlighting the ad hoc nature of the Larmor clock theory.} Dutch version on page \pageref{dutch-abstract}.