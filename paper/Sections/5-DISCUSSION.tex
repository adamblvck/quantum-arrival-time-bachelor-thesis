\section{DISCUSSION}

\subsection{Arrival Time}

The Arrival Time Distributions produce very interesting patterns, which resemble aspects of band theory in the weak barrier strength region, and regular patterns in higher barrier and energy strengths. The delta barrier arrival time peaks \textit{seem} to be separated semiclassically which is expected, but a thorough theoretical treatment is in order to confirm this. For the Gaussian barrier, non-linear effects become apparent: there seems to be resonant energies at which tunnelling occurs, which is a quantum-typical effect. Higher drop heights of the quantum particle (higher energy particles), and stronger barriers, seem to produce stable and evenly spaced arrival time peaks and intensities, whereas the opposite  (low energy) region is prone to strong non-linear effect, due to tunnelling through a Gaussian barrier. In figure \ref{fig:normalPEAKS_YO} on page \pageref{fig:normalPEAKS_YO}, one may observe that the arrival time of the particles increases linearly with drop height, which is to be expected, yet the second, third, etc$\dots$ peak, tend to stay longer above the barrier than at lower energies. This too suggests resonant tunnelling effects that have a direct impact on the arrival time distribution. The intensity distribution of arrival time peaks across different parameters generally follow intuition: less arrival intensity at every subsequent peak. Asymptotic exponential decreases in PPT are observed in the strong barrier region.

\paragraph{Peak Fitting.}

Some arrival time distributions (particularly those with strong barriers, and high drop heights) seem to produce linearly fittable peaks, yet at weaker regimes, are clearly less linear. The capacity for characterization of \textbf{arrival time peaks} using an exponentially decaying function would imply that $\tau = \frac{1}{\lambda}$ characterizes the \textbf{mean dwelling time} of the wave function above the barrier, since the time of arrival $\tau_{detection}$ is composed of $\tau_{free_1} + \tau_{tunnel} + \tau_{free_2}$. However, when considering figure \ref{fig:peak_barrier_vs_j}, it's clear that fitted peaks ought to be considered more carefully: some peaks might be first order peaks (reflections from the positive side of the barrier), and second order peaks (reflections from the negative side of the barrier), which probably reflects in a spectrum of k-values, producing the observed non-linear effects, and very chaotic fitted $\frac{1}{\lambda}$ values (see error margins). The convergence of the fitted $\tau = \frac{1}{\lambda}$ in the strong barrier regions suggests a particle that keeps on dwelling above the barrier, never arriving at the detector at $y=-L$.

\subsection{Larmor Clock vs Bohmian Tunnelling Time}

Bohmian tunnelling time defines an exact average time a particle passes through a tunnel, a notion that is rejected in canonical quantum formalism, seems to coincide almost exactly with the \textbf{Larmor Clock} theory when the barrier is well-defined. Gaussian barriers, however, are non-zero across their whole domain, making it impossible to define when a barrier begins, or when a wave function enters the barrier, since both are non-zero everywhere. Büttiker further argued that the Larmor Clock only works in weak magnetic fields, which has been confirmed in produced numerical methods under free-falling conditions. Where the weak field regime fails, Bohmian mechanics provides a straightforward and stable answer. It's experimentally impossible to measure whether a particle followed a Bohmian path or not, since the formalism is bound to measurement outcomes either way. Büttiker demonstrated theoretically that the Larmor Clock works for symmetrical non-uniform barriers, like Gaussians, which would imply that under Bohmian mechanical analysis, a Gaussian barrier has an effect radius $2R_{eff}=2.51 \sigma$, with $\sigma$ being the standard deviation of a Gaussian laser profile, although more numerical experiments are in order for different barrier widths, shapes, and particle drop heights. Fundamentally speaking, the \textbf{Larmor Clock} is an ad hoc measure of time from fundamental quantities like spin, making the Larmor Clock a derivative quantity and ultimately debatable in its true meaning, as was also suggested by Sokolovski in \cite{Sokolovski2021}.