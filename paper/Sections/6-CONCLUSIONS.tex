\section{CONCLUSION}

Bohmian mechanical treatment of arrival time distributions for free-falling particles across opaque barriers have been successfully simulated using the Split Operator Method and its fourth order extension. The arrival time distribution has been obtained, giving interesting and non-trivial quantum-typical results like suggestions of band formation, tunnelling resonances, and quasi-stable standing wave formations. Tunnelling times using Larmor Clocks and Bohmian theory produce the same values when the boundary is well-defined in weak magnetic fields. For Gaussian profiles, tunnelling times do not coincide between the theories, due to lack of well-defined boundaries, and possible category errors when fundamentally considering the classically defined \textbf{Larmor Spin Precession} in a quantum mechanical context.
\paragraph{Outlook.} The first numerical analysis of the proposed experiment has been conducted, and experimentalists can now engage with the setup and provide invaluable data to engage in the fundamental study of arrival time, and the investigation of gravitational linearity at the quantum scale. Further, a proper theoretical treatment is in order, classifying, characterizing, and perhaps even properly predicting and explaining the arrival peak times observed in the simulations done in this study.