\section*{ABSTRACT - DUTCH VERSION}
\label{dutch-abstract}

\textbf{Canonieke kwantummechanica richt zich op meetuitkomsten, terwijl alternatieve formuleringen zoals de Bohmse mechanica het volledige traject van een deeltje van A naar B beschrijven en toch identieke voorspellingen opleveren. In experimenten zoals het dubbel-spleetexperiment ontbreekt echter een nauwkeurige definitie van ‘reistijd’ of ‘aankomsttijd’ wegens problemen met de beginvoorwaarden. Vrije-valexperimenten met kwantumdeeltjes op niet-klassieke barrières kunnen dit vraagstuk mogelijk oplossen. Dit onderzoek introduceert een analyse van dergelijke opstellingen en berekent aankomsttijden via Bohmse kans stromen op de locatie van de detector. De resulterende verdelingen tonen bandvorming, tunnelresonanties en andere quantum kenmerken, wat uitnodigt tot verder theoretische behandelingen, en experimenteel onderzoek. Daarnaast worden de tunneltijden die de Larmor-klok voor deeltjes met spin $\frac{1}{2}$ voorspelt vergeleken met die uit de Bohmse mechanica. Beide theorieën komen overeen voor vierkante barrières, maar vertonen verschillen bij Gaussische barrières, een aanwijzing voor het ad hoc karakter van de Larmor-kloktheorie.}